\documentclass[11pt]{article}
\usepackage[utf8]{inputenc}
\usepackage{amstext}
\usepackage{amsmath}
\usepackage{indentfirst}
\usepackage{amssymb}

\title{PROJETO INTEGRADOR\\INTELIGENCIA ATRIFICIAL}
\author{João Mario Motidome Barradas\\Leandro Borges de Moura}

\date{10, Junho, 2022}

\begin{document}

\maketitle

\paragraph{Proposta:}

\paragraph{}

Para esta ADO, o grupo deverá criar um modelo que aprende a classificar imagens a partir de um dataset. O modelo deve ser incluído em um bot do Discord e subir em uma hospedagem gratuita como Heroku. Os alunos poderão escolher um dos datasets abaixo. Não serão permitidos outros datasets.


\paragraph{Treinamento do modelo:}

\paragraph{}

A arquitetura de rede escolhida para o trabalho é uma rede neural de multicamadas. As imagens usadas para treinar foram reudizas a 30px por 30px. Os hiperparâmetros foram escolhidos através da análise de perda e ganho observadas nos graficos gerados pelo TensorBoard. O modelo pode ser salvo usando a função model.save que gera um arquivo .keras ou .h5 que poderá ser carregado e utilizado posteiormente.


\paragraph{Uso do modelo:}

\paragraph{}

O programa carrega o modelo gerado pelo treinamento e usa a funcão predict para gerar uma array com os valores atribuidos a cada categoria. O maior valor será o o valor de certeza que o modelo tem sobre a imagem. Para retornar ao usuário o nome da categoria reconhecida usamos a funcao np.argmax na array que retorna do predict para pegar o indice do maior valor, em seguida usamos esse indice na array que contem os nomes das categorias, retornando para o usuário apenas a string que contém o nome da categoria.


\end{document}
